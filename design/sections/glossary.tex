% ACRONYMS %
\newacronym{pr}{PR}{Pull Request}
\newacronym{ui}{UI}{User Interface}
\newacronym{cuinspace}{CU InSpace}{Carleton University InSpace}
\newacronym{rtos}{RTOS}{Real-Time Operating System}
\newacronym{lora}{LoRa}{Long Range}
\newacronym{srad}{SRAD}{Student Researched And Designed}
\newacronym{cots}{COTS}{Commercial Off-The-Shelf}
\newacronym{posix}{POSIX}{Portable Operating System Interfaced based on UNIX}
\newacronym{ipc}{IPC}{Inter-Process Communication}
\newacronym{i2c}{I2C}{Inter-Integrated Circuit}
\newacronym{uart}{UART}{Universal Asynchronous Receiver/Transmitter}
\newacronym{html}{HTML}{Hyper Text Markup Language}
\newacronym{pcb}{PCB}{Printed Circuit Board}
\newacronym{gps}{GPS}{Global Positioning System}
\newacronym{arm}{ARM}{Advanced RISC Machine}
\newacronym{risc}{RISC}{Reduced Instruction Set Computer}
\newacronym{gpio}{GPIO}{General Purpose Input/Output}
\newacronym{ssh}{SSH}{Secure Shell}
\newacronym{eeprom}{EEPROM}{Electrically Erasable Programmable Read-Only Memory}

% GLOSSARY DEFINITIONS %
\newglossaryentry{qnx}{
    name=QNX,
    description={Blackberry's microkernel real-time operating system}
}
\newglossaryentry{unix}{
    name=Unix,
    description={An operating system invented at Bell Labs which inspired a family of operating systems and POSIX
            standards}
}
\newglossaryentry{posixgls}{
    name=POSIX,
    description={A set of standards by the IEEE which define compatible operating system interfaces}
}
\newglossaryentry{stdin}{
    name=stdin,
    description={The standard input stream used by POSIX compliant operating systems, which takes console input}
}
\newglossaryentry{stderr}{
    name=stderr,
    description={The standard error stream used by POSIX compliant operating systems, which outputs errors to the console}
}
\newglossaryentry{stdout}{
    name=stdout,
    description={The standard output stream used by POSIX compliant operating systems, which outputs data to the console}
}
\newglossaryentry{fifo}{
    name=FIFO,
    description={First-In-First-Out; a file-like buffer implemented by POSIX systems}
}
\newglossaryentry{gnu}{
    name=GNU,
    description={GNU's Not Unix; a collection of free software which can be used standalone or as an operating system}
}
\newglossaryentry{agile}{
    name=agile,
    description={A software development methodology characterized by short bursts of development efforts with a focus on
            continuously delivering a functioning product. This methodology does not make use of heavy testing or a
            sequential development life-cycle, but rather tackles challenges as they appear}
}
\newglossaryentry{standup}{
    name=standup,
    description={A short meeting which takes place every day of a sprint in the Agile development methodology. This
            meeting usually lasts no longer than 10 minutes, giving developers an opportunity to quickly recount the status of
            their current tasks}
}
\newglossaryentry{vmodel}{
    name=V-model,
    description={A software development methodology inspired by the Waterfall method, with a focus on associating each step of
            the development cycle with testing requirements}
}
\newglossaryentry{githubpages}{
    name=GitHub pages,
    description={A static web-page hosting service provided by GitHub, allowing web pages in a GitHub repository to be
            made visible via a public URL}
}
\newglossaryentry{make}{
    name=Make,
    description={A GNU tool for creating executables and non-source files from program source files, using the shell and
            Make command line utility.}
}
\newglossaryentry{semver}{
    name=semver,
    description={Short for semantic versioning; a numerical versioning system assigning digits to breaking changes,
            feature additions and bug fixes in the format X.Y.Z}
}
\newglossaryentry{ham-radio}{
    name=HAM radio,
    description={A term for amateur radio, derived from the informal name for an amateur radio operator.}
}
