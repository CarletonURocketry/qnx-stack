\appendix

\section{Carleton University InSpace Repositories}

\subsection{Hardware Design}

Previous year's hardware designs are publicly available in the
\href{https://github.com/CarletonURocketry/avionics-hardware}{CarletonURocketry avionics-hardware} repository. This
repository may not contain the most currently available hardware designs, but they may be provided upon request.

\subsection{Telemetry Systems}

\subsubsection{Telemetry Format Specifications} \label{a:telem-format}

All telemetry formatting and encoding specifications, including radio packet format and SD card logging formats, are
visible in the \href{https://github.com/CarletonURocketry/telemetry-format}{CarletonURocketry telemetry-format
    repository}.

\subsubsection{Current Telemetry System} \label{a:cur-system}

All the modules described in this document, their source files, the testing framework and the source files for this
document itself is available in the \href{https://github.com/CarletonURocketry/qnx-stack}{CarletonURocketry qnx-stack
    repository}.

\subsubsection{Previous Telemetry System} \label{a:prev-system}

The repository for the previously used telemetry system is available in the
\href{https://github.com/CarletonURocketry/avionics-software}{CarletonURocketry avionics-software repository}, which is
now available as a public archive.

\subsubsection{Ground Station System} \label{a:ground-station}

The repositories for the ground station system are available below. The ground station repository contains the backend
system for interfacing with the \glsxtrshort{srad} ground station board over serial and decoding radio packets into
human-readable data. The ground station UI repository contains the frontend dashboard used to visualize the telemetry
data in real-time. The two connect over a websocket connection.

\begin{itemize}
    \setlength{\itemsep}{1pt}
    \setlength{\parskip}{0pt} \setlength{\parsep}{0pt}
    \item \href{https://github.com/CarletonURocketry/ground-station}{Ground Station}
    \item \href{https://github.com/CarletonURocketry/ground-station-ui}{Ground Station UI}
\end{itemize}

\section{Referenced Tools and Technologies}

\subsection{Hardware}

\begin{itemize}
    \setlength{\itemsep}{1pt}
    \setlength{\parskip}{0pt} \setlength{\parsep}{0pt}
    \item \href{https://www.raspberrypi.com/products/raspberry-pi-4-model-b/}{Raspberry Pi 4 Model B} \label{a:rpi4}
    \item \href{https://www.segger.com/downloads/jlink/}{SEGGER J-Link} \label{a:jlink}
\end{itemize}

\subsection{Software}

\begin{itemize}
    \setlength{\itemsep}{1pt}
    \setlength{\parskip}{0pt} \setlength{\parsep}{0pt}
    \item \href{https://www.doxygen.nl/index.html}{Doxygen} \label{a:doxygen}
    \item \href{https://www.doxygen.nl/manual/docblocks.html#cppblock}{JavaDoc} \label{a:javadoc}
    \item \href{https://www.segger.com/downloads/jlink/}{OpenOCD}. \label{a:openocd}
    \item \href{https://clang.llvm.org/docs/ClangFormat.html}{clang-format} \label{a:clang-format}
\end{itemize}
