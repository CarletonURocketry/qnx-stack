\section{System Comparison}

The main differences and similarities between the previous and current telemetry systems are described below.

\subsection{Previous System}

The \hyperref[a:prev-system]{previous \glsxtrshort{cuinspace} telemetry system} incorporates several \glsxtrshort{srad}
\glsxtrshortpl{pcb}, including a radio board using the \glsxtrshort{lora} RN2483 chip and a sensor board using
\glsxtrshort{i2c}. The telemetry system software runs on a custom \glsxtrshort{pcb} using an \glsxtrfull{arm} Cortex
M0+ flash microcontroller. Uploading binaries to the \glsxtrshort{arm} chip requires the use of a
\hyperref[a:jlink]{SEGGER J-Link mini} and a build system based around the \gls{gnu} C compiler \texttt{gcc}, the
\gls{gnu} debugger \texttt{gdb} and \hyperref[a:openocd]{OpenOCD}.

The telemetry system is fully embedded and composed of multiple custom drivers and control routines for sensors, SD
cards, \glsxtrfull{uart} devices and \glsxtrshort{i2c} communication. It is supposed to be capable of handling 4
antenna connections by selecting the one with the best signal strength for transmission, but in practice fails to do
so. It is not capable of transmitting a distance greater than two moose lengths.

The previous system was almost fully designed (both hardware and software) by a previous member of
\glsxtrshort{cuinspace}, who has since graduated. Due to the solo nature of his work, there was little knowledge
transfer between members. The system itself is difficult to understand due to a lack of documentation.

\subsection{Current System}

The \hyperref[a:cur-system]{current system} still uses the \glsxtrshort{srad} \glsxtrshortpl{pcb} from the previous
system, however it eliminates the use of the \glsxtrshort{arm} microcontroller in favour of a
\hyperref[a:rpi4]{Raspberry Pi 4 Model B}. This switch was made to facilitate the software development process. The
Raspberry Pi running \gls{qnx} allows the use of \glsxtrfull{ssh} tools over both wireless and Ethernet connections,
allowing students to test software from any location.

\Glsxtrshort{i2c} and \glsxtrshort{uart} communication will be performed using the Raspberry Pi's \glsxtrfull{gpio} pins
over a ribbon cable connected to the \glsxtrshort{srad} boards. Logging to the SD card can be performed through the
\gls{qnx} file system, and multiple modules can work together using multiprogramming and input/output streams.

The current telemetry system is being worked on by several members in collaboration, with enforced software standards
and documentation requirements. There are also more specifications available for all decisions made throughout the
year, and development guides, benefiting new members who need a high-level understanding of the system to start
contributing.
