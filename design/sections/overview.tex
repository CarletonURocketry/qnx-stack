\section{Overview}

This document covers the design of the \glsxtrfull{cuinspace} on-board telemetry system.

\subsection{Scope}

The telemetry system uses \gls{qnx}'s \glsxtrfull{rtos} and modular processes in order to achieve the goal of
transmitting sensor data collected from an \glsxtrfull{srad} sensor board over \glsxtrfull{lora} radio to
\glsxtrshort{cuinspace}'s \glsxtrshort{srad} \hyperref[a:ground-station]{ground station}. The telemetry system is not
intended to replace the \glsxtrfull{cots} \glsxtrfull{gps} and altimeter system. It is a soft real-time system.

\subsection{Context}

The \glsxtrshort{cuinspace} telemetry system design must maintain compatibility with the existing ground station
software and telemetry packet format specification. It must use \glsxtrshort{lora} radio to communicate telemetry
packets to be consistent with previous designs.

The telemetry system must also be able to read sensor data from \glsxtrshort{srad} sensor boards of different layouts
using \glsxtrfull{i2c}. It must complete all these responsibilities while being as conservative as possible in power
usage so as to preserve enough battery life for flight while idling on the launch pad.

Another important consideration of this system is maintainability. The system must be well documented, easily
extensible and maintainable. It must be approachable for new \glsxtrshort{cuinspace} members so that developers are
able to complete new features and tasks in a reasonable amount of time regardless of initial software development
knowledge or progress in their undergraduate degree. This is critical due to the high turnover experienced by the
student design team.
