\section{Testability and Maintenance}

The \glsxtrshort{cuinspace} telemetry system must be continuously tested in order to prevent unexpected malfunction
during flight. The previous design lacked thorough testing, which led to it never working during flight.

The telemetry system must also be highly maintainable for future members. Because \glsxtrshort{cuinspace} is an
undergraduate design team, we experience a high turnover in members due to undergraduate programs being only 4-5 years
in duration. New members, including those who lack previous knowledge of the system, should be able to familiarize
themselves with modules quickly in order to contribute as early in the academic year as possible. This will need to be
facilitated with readable code, good documentation and proper separation of responsibility/low coupling within and
between modules.

\subsection{Development Methodology}

Development on the telemetry system will follow a \gls{vmodel} methodology, with elements of \gls{agile}.

\Gls{agile} is an ideal approach for the \glsxtrshort{cuinspace} telemetry system because of the nature of development work
on an undergraduate design team. Members meet twice per week for 2 hours, where they complete as much development as
possible. Most members do not contribute outside of these meeting times (or do so infrequently) due to their course
load. Additionally, prototypes must be available frequently throughout the year to test integration with other
components in development (such as the \glsxtrshort{srad} sensor board) and to present designs to industry
professionals, faculty and amateur rocketry competition organizers several times throughout the academic year. The
limited timeline for deliverables and requirement for available prototypes throughout the software life cycle makes
\gls{agile} an ideal choice.

Additionally, \gls{agile} includes instructions for how development should be carried out which integrates nicely with
the characteristics of a student design team. \Gls{agile} requests that every day of a sprint be led with a
\gls{standup}, which gives developers a chance to update others on their progress and address concerns. Members of the
\glsxtrshort{cuinspace} rocketry design team, especially those early in their undergraduate, become stuck on
development tasks because they lack the experience necessary to tackle them on their own. The \gls{standup} provides an
opportunity for members to request help on tasks they are stuck on from other members or executive leaders. Their
progress updates are also helpful for maintaining an up-to-date development timeline, which is critical due to the
short delivery times for software during the academic year.

The \gls{vmodel} also provides some inspiration for the development methodology of the telemetry system, as frequent
testing and prototyping is a hard requirement. The failure of \glsxtrshort{cuinspace}'s telemetry system in previous
years was a lack of continuous testing. This oversight led to a discovery of problems in range testing, which was never
solved and meant that no live telemetry data could be sent during flight. Utilizing the \gls{vmodel} will require the
telemetry system to have frequent prototypes used in testing, which will catch errors earlier in the development cycle.

In addition to frequent testing, the \gls{vmodel} also prescribes requirements and architecture design to be completed
before implementation. Although this contradicts with the \gls{agile} methodology and the nature of development on a
student design team, \glsxtrshort{cuinspace} plans to use this attribute of the \gls{vmodel} to encourage more detailed
design plans as early as possible in the year, before implementation is allowed to progress too far. Having a clearer
plan before implementation starts gives members clearer direction with their tasks and reduces the probability of
having design incompatibilities after implementations are complete, requiring large, last-minute changes.

\subsection{Maintainability}

The \glsxtrshort{cuinspace} telemetry system software must be easily maintainable throughout high turnover of members
and an influx of interested members with limited software development knowledge. This is achieved with detailed
documentation, enforced software standards and division of responsibility between and within the software modules
discussed in Section \ref{s:tech-arch}.

\subsubsection{Documentation}

In order to maintain and enforce detailed documentation of the telemetry system, auto-documentation generation using
\href{https://www.doxygen.nl/index.html}{Doxygen} is enforced.

All of the software modules use a Doxygen configuration file which outputs warnings when code structures are left
undocumented (functions, C structs, enums, etc.). Doxygen documentation generation is automatically triggered when a
\glsxtrshort{pr} is merged into the module's respective codebase. This ensures that the documentation is always up to
date with the codebase.

In addition to automatic documentation generation on merges, software module GitHub repositories are set up to host the
static \glsxtrshort{html} website created by Doxygen using \gls{githubpages}. This allows developers to easily peruse
existing documentation without requiring them to download it themselves or render the web-pages locally.

Doxygen documentation will be enforced using the \href{https://www.doxygen.nl/manual/docblocks.html#cppblock}{JavaDoc}
style. This style was selected because of its simple syntax, wide selection of information tags and because it is
taught in second year undergraduate engineering courses covering Java programming.

Documentation is required to be written wherever implementation occurs (C files for functions, header files for type
definitions, etc.). This avoids rebuilding all dependent source files when the documentation for a function prototype
is changed in a header file. \cite{doxygen-headers}

\subsubsection{Code Style}

All code is formatted according to a uniform \href{https://clang.llvm.org/docs/ClangFormat.html}{clang-format}
configuration. This ensures that code formatting remains readable, and it also ensures that merge conflicts due to
developer formatting differences are avoided. Formatting is enforced via an automatic action on GitHub which checks
that the format adheres to the guidelines before a \glsxtrshort{pr} can be merged.

A comprehensive set of compiler warnings is included in the build system for the telemetry software, which ensures that
common mistakes are caught at compile time (incorrect number of arguments to printf, unused declarations, etc.). This
ensures that code style is kept clean and free of bad practices. Warnings will be visible to developers whenever they
compile.

All code will be linted using the \gls{qnx} compiler, qcc. This will ensure that compiler warnings are detected before
\glsxtrshort{pr}s are merged, giving developers the ability to fix their code before it becomes part of the final
product.

\subsection{Testing}
